\usepackage{enumerate}

% Comandos personalizados

% Cabeçalhos 
\newcommand{\cabeca}[4]{
	\begin{tabular}{cc}
		\multirow{3} * {\includegraphics[scale=1.5]{#1}} & \Large #2 \\  
		& \Large #3 \\ 
		& \Large #4 \\ 
	\end{tabular} 
	\vspace{2cm}
} 

%items da identificação
\newcommand{\meuitem}[2]{
	\item \Large \textbf{#1:} #2
}

% Lista de imagens e fotos.
\newcommand{\listafotos}[2]{
	\Large \textbf{#1: #2}\\
}

% inserção de fotografias
\newcommand{\foto}[5]{
	\begin{figure}[!htb]
		\centering
		\caption{\textbf{#1}}
		\includegraphics[scale=#2]{#3}\\
		{\footnotesize Fonte: #4}
		\label{#5}
	\end{figure}
}

% Contadores 
%para número de questões
\newcounter{questao}
\setcounter{questao}{1}

% Para o número de definições
\newcounter{definicao}
\setcounter{definicao}{1}
%------------------------------------------------------

%Comandos automatizados 

%para questões
\newcommand{\quest}{
	\noindent \textbf{Questão \thequestao: \stepcounter{questao}}}

% Para definições
\newcommand{\definicao}[2]{
	\noindent\textbf{Definição \thedefinicao:}\\ \stepcounter{definicao}
	\textit{\textbf{#1} #2}
}
%--------------------------------------------------------------------------------

% Ajustes dos ambientes matemáticos
\everymath{\displaystyle}

%teste

\newcommand{\teste}{
	\textbf{Funcionou!}
}

\newcommand{\optabertas}[4]{
	\vspace{0.25cm}
	\begin{enumerate}[(a)]
		\item #1
		\vspace{1cm}
		\item #2
		\vspace{1cm}
		\item #3
		\vspace{1cm}
		\item #4
		\vspace{1cm}
	\end{enumerate}
}