% Alguns comandos que podem facilitar a criação de algumas estruturas
% Autor: Diógenes Santana da Silva
% Data: 09/08/2023

% Para fazer uso destes comandos
% basta fazer o download deste ar-
% quivo e, no preâmbulo de seu códi-
% go principal chamá-lo com o coman-do
% "\input{}"

\usepackage{enumerate}

% Comandos personalizados

% Cabeçalhos 
\newcommand{\cabeca}[4]{
	\begin{tabular}{cc}
		\multirow{3} * {\includegraphics[scale=1.5]{#1}} & \Large #2 \\  
		& \Large #3 \\ 
		& \Large #4 \\ 
	\end{tabular} 
	\vspace{2cm}
} 

%items da identificação
\newcommand{\meuitem}[2]{
	\item \Large \textbf{#1:} #2
}

% Lista de imagens e fotos.
\newcommand{\listafotos}[2]{
	\Large \textbf{#1: #2}\\
}

% inserção de fotografias
\newcommand{\foto}[5]{
	\begin{figure}[!htb]
		\centering
		\caption{\textbf{#1}}
		\includegraphics[scale=#2]{#3}\\
		{\footnotesize Fonte: #4}
		\label{#5}
	\end{figure}
}

% Contadores 
%para número de questões
\newcounter{questao}
\setcounter{questao}{1}

% Para o número de definições
\newcounter{definicao}
\setcounter{definicao}{1}
%------------------------------------------------------

%Comandos automatizados 

%para questões
\newcommand{\quest}{
	\noindent \textbf{Questão \thequestao: \stepcounter{questao}}}

% Para definições
\newcommand{\definicao}[2]{
	\noindent\textbf{Definição \thedefinicao:}\\ \stepcounter{definicao}
	\textit{\textbf{#1} #2}
}
%--------------------------------------------------------------------------------

% Ajustes dos ambientes matemáticos
\everymath{\displaystyle}

%teste

\newcommand{\teste}{
	\textbf{Funcionou!}
}

\newcommand{\optabertas}[4]{
	\vspace{0.25cm}
	\begin{enumerate}[(a)]
		\item #1 
		\vspace{1cm}
		\item #2
		\vspace{1cm}
		\item #3
		\vspace{1cm}
		\item #4
		\vspace{1cm}
	\end{enumerate}
}

%Diagramas

%setas
\newcommand{\setas}{
\begin{center}
	\begin{tikzpicture}
		% Domínio 
		\node (2) at (0,-1) {2}; \filldraw (2.east) circle (1pt);
		\node (5) [below of = 2] {5}; \filldraw (5.east) circle (1pt);
		\node (11) [below of = 5] {11}; \filldraw (11.east) circle (1pt);
		\node (13) [below of = 11] {13}; \filldraw (13.east) circle (1pt);
		\node[fit=(2) (5) (11) (13), ellipse, draw=red, minimum width=2cm, thick, label=below:\(A\)]{};
		
		% Contradomínio
		\node (4) at (3.5, 0) {4}; \filldraw (4.west) circle (1pt);
		\node (10) [below of = 4] {10}; \filldraw (10.west) circle (1pt);
		\node (22) [below of = 10] {22}; \filldraw (22.west) circle (1pt);
		\node (26) [below of = 22] {26}; \filldraw (26.west) circle (1pt);
		\node (70) [below of = 26] {70}; \filldraw (70.west) circle (1pt);
		\node (35) [below of = 70] {35}; \filldraw (35.west) circle (1pt);
		\node[fit=(4) (10) (22) (26) (70) (35), ellipse, draw=blue, minimum width=2cm, thick, label=below:\(B\)]{};
		\node[fit=(4) (10) (22) (26), sqrt, draw=black, minimum width=1cm, thick]{};
		
		% Setas
		\draw [->, shorten >=.1cm, >=stealth] (2.east) to [out=10 in=120] (4.west);
		\draw [->, shorten >=.1cm, >=stealth] (5.east) to [out=10 in=120] (10.west);
		\draw [->, shorten >=.1cm, >=stealth] (11.east) to [out=10 in=120] (22.west);
		\draw [->, shorten >=.1cm, >=stealth] (13.east) to [out=10 in=120] (26.west);
	\end{tikzpicture}
\end{center}
}

